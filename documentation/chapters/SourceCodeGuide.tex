\chapter{Source code guide}
\label{chap:Source code guide}

\section{Header files}
\label{sec:Header files}

The header files ({\tt *.h}) are located in {\tt src/include}. In {\tt DNA-functions.h}, the function prototypes are defined. The file {\tt DNA-constants.h} contains macros of constants (such as $\pi$) and mathematical operators (such as trigonometric functions or $\mathrm{min/max}$ operators). In the {\tt DNA.h} file, all structures used throughout the program are declared. There are four main structures, the instances of which are created in {\tt main.c}:

\begin{compactitem}
\item {\tt struct DNA\_RunOptions}: Structure containing variables and function pointers associated with the run options. Nested structures:
\begin{compactitem}
\item {\tt struct DNA\_NumericsFD}: Structure containing variables used for the temporal and spatial discretization (e.g.~$\Delta t$, $\Delta x$, etc.), and function pointers associated with the boundary conditions. Nested structure: {\tt struct DNA\_FDCoeffs} that holds the finite-difference coefficients.
\item {\tt struct DNA\_WaveExcitation}: Variables of the wave excitation functions, see Eqs.~\eqref{eq:ExcitationSine} and \eqref{eq:GaussEnvelope}.
\item {\tt struct DNA\_Probes}: Variables specifying the acoustic pressure probes (see Section \ref{sec:Run options and fluid properties}).
\end{compactitem}
\item {\tt struct DNA\_FluidProperties}: Fluid properties, such as the density, speed of sound, and fluid nonlinearity.
\item {\tt struct DNA\_Fields}: Structure comprising all scalar fields. Nested structures:
\begin{compactitem}
\item {\tt struct DNA\_Grid}: Scalars and scalar fields, represented by pointers, representing the dependent coordinates $x\left(\xi,t\right)$ of the time-dependent physical domain $\Omega\left(t\right)$, the fixed computational domain $\Theta$, and the metrics of the coordinate transformation according to Eqs.~\eqref{eq:linJacobian} to \eqref{eq:lindqdt}.
\item {\tt struct DNA\_BackgroundFlowField}: Scalar fields, represented by pointers, involving the background flow field, as well as spatial and temporal derivatives thereof.
\item {\tt struct DNA\_PhiField}: Scalar fields, represented by pointers, involving the acoustic potential (the primary solution variable), as well as spatial and temporal derivatives thereof. The acoustic pressure is included in this structure as well.
\item {\tt struct DNA\_OldPhiField}: Similar to the previous structure, but containing the data from the two preceding time steps. This information is required for the time integration.
\end{compactitem}
\item {\tt struct DNA\_MovingBoundary}: Variables associated with the moving boundary of the physical domain $\Omega\left(t\right)$.
\end{compactitem}



\section{Source files}
\label{sec:Source files}

The core of the computational method revolves around solving the lossless convective Kuznetsov equation \citep{Schenke_et_al_2023_JASA} in the coordinates of the fixed computational domain. The following table provides an overview and brief description of the source files.

\noindent
\begin{longtable}{p{0.38\textwidth} p{0.57\textwidth}}
\textbf{Source file} & \textbf{Description}
\vspace{1mm} \\
\hline Directory {\tt backgroundflow} &\\ \hline
{\tt backgroundflowgravitation.c} & Function to compute the gravitational potential, if applicable.\\
{\tt backgroundflowmotion.c} & Functions to compute the background flow velocity $u_0$ and its spatial and temporal derivatives. \\
\\
\hline Directory {\tt boundary} &\\ \hline
{\tt boundaryconditions.c} & Calling the functions to set the boundary conditions. \\
{\tt boundarymotion.c} & Functions to describe the motion of the moving domain boundary. \\
\\
\hline Directory {\tt fd} &\\ \hline
{\tt fdboundaryconditions.c} & Functions to apply the boundary conditions. The functions for the \textit{west} and \textit{east} boundaries are called by reference. The corresponding pointers are {\tt int (*FDBC\_West)(...)} and {\tt int (*FDBC\_East)(...)}, respectively.  \\
{\tt fdcoeffs.c} & Finite difference coefficients for the spatial and temporal discretization. \\
{\tt fddt.c} & Functions to construct the explicit finite differences of the time derivatives. \\
{\tt fddx.c} & Functions to construct the explicit finite differences of the spatial derivatives.  \\
{\tt fdsumaphi.c} & Functions to compute the sums over previous time-steps needed for the finite-difference discretization, see Eq.~\eqref{eq:fddt_compact}. \\
\\
\hline Directory {\tt grid} &\\ \hline
{\tt gridmakegrid.c} & Functions to create the time-dependent grid for the physical domain $\Omega\left(t\right)$ and the fixed grid for the computational domain $\Theta$. \\
{\tt gridmotion.c} & Updating the derivatives of the dependent coordinate $\xi\left(r,t\right)$ as given by Eqs.~\eqref{eq:linJacobian} to \eqref{eq:lindqdt}. \\
\\
\hline Directory {\tt include} &\\ \hline
{\tt DNA.h} & See Section \ref{sec:Header files}. \\
{\tt DNA-constants.h} & See Section \ref{sec:Header files}. \\
{\tt DNA-functions.h} & See Section \ref{sec:Header files}. \\
\\
\hline Directory {\tt initialize} &\\ \hline
{\tt initializefield.c} & Routines to initialize single fields of the type {\tt struct DNA\_Fields}. \\
{\tt initializeprocessoptions.c} & Setting all function pointers based on the specifications in the {\tt Wave-DNA} options file or the default options. \\
{\tt initializesimulation.c} & Routine that allocates memory for all scalar fields and results vectors by calling functions in {\tt memoryallocfields.c} (see below). Initial values are set and the grid point IDs corresponding to the coordinates of the probe sample locations (IDs of the nearest neighbours are taken) specified in the {\tt Wave-DNA} options file are identified. \\
\\
\hline Directory {\tt io} &\\ \hline
{\tt iodefaultoptions.c} & Defaults for the options that can be set in the {\tt Wave-DNA} options file. A simulation is performed even with an empty options file (see Section \ref{sec:defaultrun}). \\
{\tt ioonscreen.c} & Functions for output on screen during the run time of the simulation. \\
{\tt ioreadcommandlineoptions.c} & Function to read command line options. \\
{\tt ioreadoneoption.c} & Called in {\tt ioreadoptionsfile.c} to read a single option the {\tt Wave-DNA} options file. \\
{\tt ioreadoptionsfile.c} & Function that reads the {\tt Wave-DNA} options file. \\
{\tt ioresults.c} & Functions to write output data to disk. \\
\\
\hline Directory {\tt memory} &\\ \hline
{\tt memoryalloc.c} & Function to allocate memory for fields of the type {\tt struct DNA\_Field} and the probe sample points.\\
{\tt memoryallocfields.c} & Routine calling {\tt memoryalloc.c} to allocate all fields. \\
{\tt memoryfreefields.c} & Routine called in {\tt main.c} to free all fields upon termination of the program. \\
\\
\hline Directory {\tt solve} &\\ \hline
{\tt solveexplicitderivatives.c} & Routine calling the functions in {\tt fddx.c} and {\tt fddt.c} to construct the finite-difference approximations of the explicit temporal, spatial, and mixed derivatives. \\
{\tt solveloop.c} & This file contains the function {\tt SolveTimeLoop}, representing the time loop of the simulation, and the function {\tt Solve}, which is called in {\tt {\tt SolveTimeLoop}} at each time step and which calls the functions relevant to the numerical algorithm. \\
{\tt solvepredictorcorrector.c} & Routines to apply the predictor (see Eq.~\eqref{eq:discreteEqn}) and the corrector steps (see Eq.~\eqref{eq:corrector}). Additional routines are to update the Laplacian prior to the corrector step and to store the initial guess $\widetilde{\Phi_{1,i}}$ computed based on Eq.~\eqref{eq:discreteEqn} and required in Eq.~\eqref{eq:corrector}. \\
{\tt solvesumaphi.c} & Routine calling the functions in {\tt fdsumaphi.c} to compute the sums over the previous time steps required for the explicit finite differences of the time derivatives. \\
{\tt solveupdatefields.c} & Contains the function that updates the old fields. \\
\\
\hline Directory {\tt transform} &\\ \hline
{\tt transformeqn.c} & Functions to compute the space- and time-dependent coefficients of the transformed wave equation, and to assemble the terms of the discretized equation. The file contains the function {\tt TransformEqn\_Predictor} to assemble the terms for the predictor step of the predictor-corrector method (see Eq.~\eqref{eq:discreteEqn}), and the function {\tt TransformEqn\_Corrector}, which only updates the coefficients and terms that are relevant for the corrector step, see Eq.~\eqref{eq:corrector}.\\
{\tt transformgeometricaldecay.c} & Function to account for the change of the cross-sectional area $A$ in the Laplacian in the case of spherical symmetry. \\
{\tt transformpotential.c} & Routines to convert the acoustic pressure $p_1$ into the perturbation potential $\Phi_1$ (required at the excitation node) and vice versa (required to reconstruct the acoustic pressure field from the solution for $\Phi_1$). \\
\\
\hline Directory {\tt waveexcitation} &\\ \hline
{\tt waveexcitation.c} & Functions to excite the acoustic pressure wave at the specified excitation node. \\
 \hline
\end{longtable} \vspace{1em}

