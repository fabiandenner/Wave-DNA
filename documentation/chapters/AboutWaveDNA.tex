\chapter{About Wave-DNA}
\label{chap:About Wave-DNA}

{\tt Wave-DNA} is a simulation tool for one-dimensional and spherically-symmetric nonlinear acoustic waves in transient and spatially variable background flow fields. The motion of the background medium is accounted for by considering a convective form of the lossless Kuznetsov wave equation \citep{Schenke_et_al_2023_JASA}, derived from first principles based on perturbations of the continuity equation and the transient Bernoulli equation. The background flow field in which the acoustic waves propagate is treated as an input to the wave solver. In principle, the background flow field my be obtained from analytical considerations, numerical simulations, or experimental measurements. The tool recommends itself for the fundamental study of nonlinear Doppler phenomena in acoustic wave propagation.

The moving boundary may represent a moving wave emitter or scatterer, and it may move relative to the fluid or displace the surrounding fluid. This allows the user to study nonlinear Doppler effects associated with moving wave emitters/scatterers and/or non-uniform background flow fields. The motion of the domain boundary is conveniently taken into account by a generic coordinate transformation of the governing wave equation, where the transformed wave equation is solved in a fixed computational domain. This technique enables accurate numerical solutions of the combined wave-flow problem without the necessity to interpolate data between the moving grid points in the physical domain. The numerical technique is based on explicit finite differences, and it is equipped with a predictor-corrector method to counteract the onset and growth of dispersive numerical noise in the cases of broad frequency bands or shock formation \citep{Schenke_et_al_2022}. Furthermore, absorbing boundary conditions can be imposed in order to let the acoustic waves pass the domain boundaries without reflection.

A particularly well-suited example to demonstrate the capability of the simulation framework is the so-called acoustic black/white hole analogue as presented in \citep{Schenke_et_al_2022_PoF, Schenke_et_al_2023_JASA}. The present documentation provides detailed instructions on how to conduct acoustic black and white hole simulations using {\tt Wave-DNA}, as well as other examples to demonstrate the capabilities and functionalities of the tool.

The {\tt Wave-DNA} repository is located at \href{https://github.com/polycfd/Wave-DNA}{\texttt{https://github.com/polycfd/Wave-DNA}}, is under the copyright of its developers and made available as open-source software under the terms of the \href{https://opensource.org/license/mit/}{MIT License}. Details about the theory and numerical methods of {\tt Wave-DNA}, as well as examples of its scientific applications, can be found in these papers: \citep{Schenke_et_al_2022, Schenke_et_al_2022_PoF, Schenke_et_al_2023_JASA}.
The development of {\tt Wave-DNA} has directly benefitted from research funding provided by the Deutsche Forschungsgemeinschaft (DFG, German Research Foundation), grant number 441063377.


